

 \documentclass[12pt,a4paper,finnish,oneside]{article}
 
%\usepackage[latin1]{inputenc}
\usepackage[utf8]{inputenc} 
 \usepackage[T1]{fontenc}
 \usepackage[finnish]{babel}
 \usepackage{url}


\newcommand{\fixme}[1]{\textsl{[#1]}}




\begin{document}


\title{Automaattisen nimentunnistimen kehittäminen suomen kielelle: Tutkimussuunnitelma}
\author{Teemu Ruokolainen}


\maketitle


\section{Tutkimusongelman kuvaus}

Elektronisessa muodossa olevan tekstin määrän raju kasvu viime vuosikymmeninä on lisännyt tarvetta kehittää automaattisia tapoja käsitellä ja analysoida kirjoitettua kieltä. Tähän tarkoitukseen käytettyihin tekstinlouhintamenetelmiin kuuluvat mm. juoksevassa tekstissä esiintyvien nimien tunnistamiseen käytetyt tekniikat. Tässä suunnitelmassa kuvatussa tutkimuksessa keskitytään automaattisen nimentunnistimen kehittämiseen suomen kielelle. 

%Tähän tarkoitukseen käytettyihin tekstinlouhintamenetelmiin kuuluvat mm. erilaiset dokumenttien luokittelijat ja ryhmittelijät sekä juoksevassa tekstissä esiintyvien konseptien tunnistamiseen käytetyt tekniikat. Ehdotetussa tutkimuksessa keskitytään jälkimmäisiin menetelmiin kuuluvan nimentunnistajan kehittämiseen suomen kielelle.

Nimentunnistus tähtää juoksevassa tekstissä esiintyvien termien, kuten henkilöiden (\textit{Sauli Niinistö}), paikkojen (\textit{Rovaniemi}) ja organisaatioiden (\textit{Nokia Networks}) nimien paikantamiseen ja luokitteluun. Tunnistustulosta voidaan käyttää hyväksi ihmisen tai automaattisen järjestelmän suorittamassa tiedonlouhinnassa tai esimerkiksi dokumenttitietokannan hakutoiminnoissa. Automaattinen nimentunnistus tukee siis mm. Citizen Mindscapes:in \cite{citizenmindscapes} ja Kansalliskirjaston kokoelmien digitoimisen \cite{kansalliskirjasto} kaltaisia digitaalisten ihmistieteiden tutkimushankkeita. Nimentunnistusta on myös hyödynnetty teollisuudessa mm. Uutisnenä-sovelluksessa \cite{huovelin2013}.


\section{Tutkimuksen tavoite}

Tutkimuksen tavoitteena on kehittää ensimmäinen moderneja tilastollisia menetelmiä hyödyntävä automaattinen nimentunnistin suomen kielelle. Tunnistimen toteutus julkaistaan avoimena lähdekoodina ja sitä voidaan hyödyntää vapaasti tutkimuskäytössä. Työkalun kehittämisestä julkaistaan myös vertaisarvioitu tieteellinen artikkeli esimerkiksi Journal of Language Resources and Evaluation-julkaisussa (JuFo-taso 2).


\section{Menetelmät}

Tällä hetkellä suomen kielelle on saatavilla Helsingin Yliopiston Nykykielten laitoksella kehitetty automaattinen nimentunnistin, FiNer \cite{}. FiNer hyödyntää perinteistä sääntöpohjaista tunnistustapaa, jossa nimet paikannetaan ja luokitellaan käyttäen käsin muodostettuja sääntöjä. Sääntöpohjaisen lähestymistavan vahvuus piilee sen hyvässä tarkkuudessa mutta heikkous huonossa kattavuudessa. Menetelmän hyvä tarkkuus tarkoittaa sitä, että kaikki menetelmän palauttamat termit ovat suurella todennäköisyydellä oikeita nimiä. Menetelmän huono kattavuus kuitenkin tarkoittaa sitä, että suuri osa tekstissä esiintyvistä nimistä jää tunnistamatta suurella todennäköisyydellä. Tämän vuoksi modernit nimentunnistimet, kuten alunperin englannille kehitetty Stanford Named Entity Recognizer \cite{stanfordner}, hyödyntävät tilastollisia koneoppimismenetelmiä. Toisin kuin sääntöpohjainen lähestymistapa, tilastollinen lähestymistapa mahdollistaa sekä hyvän tunnistustarkkuuden että -kattavuuden. Tilastollista lähestymistapaa hyödyntävä tunnistin voidaan myös laajentaa useammalle tekstilajille suoraviivaisemmin kuin sääntöihin perustuva menetelmä. Näistä syistä ehdotetussa tutkimuksessa keskitytään nimentunnistimen kehittämiseen hyödyntäen tilastollista koneoppimista.

Tilastollisen nimentunnistimen kehitysprosessi jakaantuu kahteen osaan. 

\begin{itemize}

\item[1)] Ihmisasiantuntija tunnistaa, eli paikantaa ja luokittelee, nimet tekstiaineistossa. Tekstiaineistona käytetään uutissivusto DigiToday:sta vuosilta 2014 ja 2015 haettuja artikkeleita. Digitodayn artikkelit kuuluvat Creative Commons-lisenssin piiriin. Tämä tarkoittaa, että myös tekstiaineisto nimentunnistusmerkintöineen voidaan julkaista tutkimusyhteisön käyttöön.

\item[2)] Kohdassa 1) luotua aineistoa käytetään tilastollisen tunnistin opettamiseen. Tunnistimen toteutuksessa hyödynnetään äskettäin suomen kielelle julkaistua tilastollista morfologista jäsennintä, FinnPos:ia \cite{silfverberg2015}\cite{finnpos}. Vapaassa levityksessä oleva FinnPos perustuu menetelmällisesti samaan laadukkaaseen oppimistekniikkaan kuin mm. yllä mainittu Stanford Named Entity Recognizer. Muodostetun opetusaineiston lisäksi tunnistimen toteutuksessa voidaan hyödyntää jo olemassaolevaa sääntöpohjaista FiNer:iä.

\end{itemize}



\section{Aikataulu, suorituspaikka ja vastuullinen johtaja}

% Hanke aloitettiin Helsingin Yliopiston Nykykielten laitoksella syyskuussa 2015. Nimentunnistimen ensimmäinen versio julkaistaan joulukuussa 2016. 
% Hanke toteutetaan yhteistyössä Miikka Silfverbergin (tutkija, Helsingin Yliopiston Nykykielten laitos) kanssa Krister Lindenin (johtava tutkija, Helsingin Yliopiston Nykykielten laitos) valvonnassa.

Tutkimushanke toteutetaan 21.2.2016 ja 28.2.2017 välisenä aikana Helsingin Yliopiston Nykykielten laitoksella Krister Lindenin (johtava tutkija, Helsingin Yliopiston Nykykielten laitos) valvonnassa.

%Hankkeen kokonaiskustannukseksi vuonna 2016 arvioidaan 36000 euroa. Summa koostuu hakijan palkkakustannuksista. FinClarin-projekti () on sitoutunut maksamaan summasta 50\% (18000 euroa). %Tässä hakemuksessa esitetty 18000 euroa muodostaa loput 50\% kokonaiskustannuksista.





%Nimentunnistaja kehitetään hyödyntäen modernia tilastollista koneoppimismetodologiaa. Kehitysprosessi jakaantuu kahteen osaan. Ensimmäisessä osassa ihmisasiantuntija paikantaa ja merkitsee nimet tekstiaineistoon. Lähtökohtaisena tekstiaineistona käytetään uutissivusto DigiTodaysta vuosina 2014 ja 2015 haettuja artikkeleita. Toisessa osassa merkittyä aineistoa käytetään tilastollisen tunnistajan opettamiseen. Ihmisen muodostaman opetusaineiston lisäksi tunnistaja hyödyntää Helsingin Yliopistossa aikasemmin kehitettyä tilastollista morfologista suomen kielen jäsennintä, FinnPosia \citep{}, sekä sääntöpohjaista suomen kielen nimentunnistajaa, FiNer:iä \citep{}. Sääntöpohjaisen FiNer-tunnistajan vahvuus piilee sen hyvässä tarkkuudessa (precision) mutta heikkous huonossa kattavuudessa (recall). Menetelmän hyvä tarkkuus merkitsee sitä, että kaikki tunnistetut termit ovat suurella todennäköisyydellä oikeita nimiä. Huono kattavuus taas merkitsee sitä, että suuri osa nimistä jää tunnistamatta suurella todennäköisyydellä. Tilastolliset menetelmät sitä vastoin saavuttavat sekä hyvän tarkkuuden että hyvän kattavuuden. Tämän vuoksi parhaimmat nimentunnistimet, kuten alunperin englannille kehitetty Stanford Named Entity Recognizer \citep{}, hyödyntävät tilastollista koneoppimismetodologiaa. Tilastollista lähestymistapaa hyödyntävää tunnistajaa on myös helpompi laajentaa muihin tekstilajeihin verrattuna sääntöpohjaiseen menetelmään.


%Automaattisen kielenkäsittelyn tutkimuksessa nimentunnistus on ollut pitkään kehityksen kohde. Esimerkiksi englannille ensimmäiset modernia tilastollista metodologiaa käyttävät tunnistustyökalut (mm. Stanford Named Entity Recognizer) julkaistiin jo 2000-luvun puolivälissä. On kuitenkin tyypillistä, että pienemmillä kielialueilla kieliresurssien kehittäminen ja julkaiseminen tapahtuu jäljessä. Esimerkiksi ensimmäinen englannin kielisiä resursseja vastaava ruotsinkielinen nimentunnistustyökalu (SweNER) julkaistiin vuonna 2013. Vastaavaa resurssia ei ole julkaistu suomelle.



\bibliographystyle{plainnat}

\begin{thebibliography}{4}

\bibitem{citizenmindscapes}
\url{https://tuhat.halvi.helsinki.fi/portal/en/projects/citizen-mindscapes-(41cc84f8-af4f-4b15-8e9d-4a8ce69daf0c).html}

\bibitem{kansalliskirjasto}
\url{http://digi.kansalliskirjasto.fi/}

\bibitem{huovelin2013}
Huovelin Juhani, Gross Oskar, Solin Otto, Linden Krister, Maisala Sami Petri Tapio, Oittinen Tero, Toivonen Hannu, Niemi Jyrki ja Silfverberg Miikka.
2013.
Software Newsroom – an approach to automation of news search and editing.
\textit{Journal of Print Media Technology research},
vol 2,
no. 3, 
s. 141-156.

\bibitem{silfverberg2015}
Silfverberg Miikka, Ruokolainen Teemu, Linden Krister, Kurimo Mikko.
(hyväksytty).
FinnPos: an open-source morphological tagging and lemmatization toolkit for Finnish.
\textit{Journal of Language Resources and Evaluation}.

\bibitem{finnpos}
\url{https://github.com/mpsilfve/FinnPos}

%\bibitem{manning2005}
%Finkel Jenny Rose, Grenager Trond ja Manning Christopher. 2005. Incorporating Non-local Information into Information Extraction Systems by Gibbs Sampling. Proceedings of the 43nd Annual Meeting of the Association for Computational Linguistics (ACL 2005), pp. 363-370.

\bibitem{stanfordner}
\url{http://nlp.stanford.edu/software/CRF-NER.shtml}

\end{thebibliography}


\end{document}


